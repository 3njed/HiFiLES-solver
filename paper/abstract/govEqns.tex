\section{Governing Equations}
\label{sec:govEq}

In this article, we are concerned with time-accurate, viscous flow around aerodynamic bodies in arbitrary motion which is governed by the compressible, unsteady Navier-Stokes (NS) equations. Consider the equations in a domain, $\Omega \subset \mathbf{R}^3$, with a disconnected boundary that is divided into a far-field component, $\Gamma_\infty$, and an adiabatic wall boundary. The surface $S$ represents the outer mold line of an aerodynamic body, and it is considered continuously differentiable ($C^1$). These conservation equations along with a generic source term, $\mathcal{Q} $, can be expressed in an arbitrary Lagrangian-Eulerian (ALE) differential form as

\begin{equation} 
\label{rans}
\left\{\begin{array} {lll}
\mathcal{R}(U) = \frac{\partial U}{\partial t} + \nabla \cdot \vec{F}^{c}_{ale} -  \nabla \cdot ( \mu_{tot}^{1}\vec{F}^{v1} + \mu_{tot}^{2}\vec{F}^{v2}) - \mathcal{Q} = 0, & \mbox{  in } \Omega, & t > 0  \\
\vec{v} = \vec{u}_\Omega, &  \mbox{ on }  S,  \\
\partial_n T = 0, &  \mbox{ on }  S,   \\
(W)_+ = W_\infty,  & \mbox{ on }  \Gamma_\infty, \\
\end{array}\right.
\end{equation}
where
\begin{align} 
\label{euler_f}
U = \left \{ \begin{array}{c} \rho \\ \rho \vec{v} \\ \rho E \end{array} \right \},
\vec{F}^{c}_{ale} = \left \{ \begin{array}{c} \rho (\vec{v} - \vec{u}_\Omega)  \\ \rho \vec{v} \otimes  (\vec{v} - \vec{u}_\Omega) + \bar{\bar{I}} p \\ \rho E (\vec{v} - \vec{u}_\Omega) + p \vec{v}   \end{array} \right \},
\vec{F}^{v1} = \left \{ \begin{array}{c} \cdot \\ \bar{\bar{\tau}} \\ \bar{\bar{\tau}} \cdot \vec{v}  \end{array} \right  \},
\vec{F}^{v2} = \left \{ \begin{array}{c} \cdot \\ \cdot \\ \ c_p \nabla T   \end{array} \right \}, 
\mathcal{Q} = \left \{ \begin{array}{c} q_{\rho} \\ \vec{q}_{\rho \vec{v}} \\ \ q_{\rho E}   \end{array} \right \},
\end{align}
%$\rho$ is the fluid density, $\vec{v} = \{v_1, v_2, v_3\}^\mathsf{T} \in \mathbb{R}^{3}$ is the flow speed in a Cartesian system of reference, $\vec{u}_\Omega$ is the velocity of a moving domain (mesh velocity after discretization), $E$ is the total energy per unit mass, $p$ is the static pressure, $c_p$ is the specific heat at constant pressure, $T$ is the temperature, and the viscous stress tensor can be written in vector notation as
\begin{align} 
\label{tau}
\bar{\bar{\tau}} = \nabla \vec{v} + {\nabla \vec{v}}^\mathsf{T}  - \frac{2}{3} \bar{\bar{I}} (\nabla \cdot \vec{v} ).
\end{align}

%The second line of Eqn.~\ref{rans} represents the no-slip condition at a solid wall, the third line represents an adiabatic condition at the wall, and the final line represents a characteristic-based boundary condition at the far-field where the fluid state at the boundary is updated using the state at infinity depending on the sign of the eigenvalues~\cite{Hirsch:Numerical}. Note that the boundary conditions take into account any domain motion. The temporal conditions will be problem dependent, and for this article, we will be interested in time-periodic flows where the initial and terminal conditions do not affect the time-averaged behavior over the time interval of interest, $\mathbb{T} = t_f - t_o$. Assuming a perfect gas with a ratio of specific heats, $\gamma$, and gas constant, $R$, the pressure is determined from
\begin{align} 
\label{constitutive}
p = (\gamma-1) \rho \left [ E - \frac{1}{2} (\vec{v} \cdot \vec{v} ) \right ],
\end{align}
 and the temperature is given by
\begin{align} 
\label{enthalpy}
T = \frac{p}{\rho R}.
\end{align}