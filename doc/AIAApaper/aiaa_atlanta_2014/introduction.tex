% !TEX root = ./main.tex

\section{Introduction}

Over the last 20 years, much fundamental work has been done in developing high-order numerical methods for Computational Fluid Dynamics. Moreover, the need to improve and simplify these methods has attracted the interest of the applied mathematics and the engineering communities. Now, these methods are beginning to prove themselves sufficiently robust, accurate, and efficient for use in real-world applications.

However, low-order numerical methods are still the standard in the aeronautical industry. There has been a long-term, sustained scientific and economical investment to develop this successful and robust technology. Currently, an industry-standard, second-order finite volume computational tool performs adequately well in a broad range of aeronautical engineering applications. For that reason, the introduction of new, high-order numerical schemes in the aeronautical industry is challenging, particularly in areas where the low-order numerical methods already provide the required robustness and accuracy (keeping in mind the limitations of current turbulence model technology).

Thanks to new and emerging aircraft roles (very small or large concepts, very high or low altitude, quiet vehicles, low fuel consumption vehicles, etc.), revolutionary aircraft design concepts will appear in the near future, and the need for high-fidelity simulation techniques to predict their performance is growing rapidly. Undoubtedly, high-order numerical methods are starting to find their place in the aeronautical industry. 

Unsteady simulations, including those of flapping wings, wake capturing, noise prediction, and turbulent flows via Large Eddy Simulation (LES), are just a few examples of computations that could benefit from high-order numerical methods. In particular, high-order methods have a significant edge in applications that require accurate resolution of the smallest scales of the flow. Such situations include the generation and propagation of acoustic noise from an airframe, or at the limits of the flight envelope where unsteady, vortex-dominated flows have a significant effect on aircraft performance. On a given grid, utilizing a high-order representation enables smaller scales to be resolved with a greater degree of accuracy than standard second-order methods. Furthermore, high-order methods are inherently less dissipative, resulting in less unwanted interference with the correct development of the turbulent energy cascade. This factor makes the combination of high-order numerics with LES modeling very powerful, with the potential to significantly improve upon the accuracy and computational cost of the standard approach of coupling LES with second-order methods. The amount of computing effort to achieve a small error tolerance can also be much smaller with high-order than second-order methods. Even real time simulations (one second of computational time, one second of real flight), could benefit from high-order algorithms that feature more intensive computation within each mesh element (ideal for vector machines and new computational platforms like GPUs, FPGAs, coprocessors, etc).

However, before claiming the future success of high-order numerical methods in industry, two main difficulties should be overcome: a) high-order numerical schemes must be as robust as state-of-the-art low-order numerical methods, b) the existing level of verification and validation (V\&V) in high-order CFD codes should be similar to the typical level of their low-order counterparts.

During the last decade, the Aerospace Computing Laboratory (ACL) of the Department of Aeronautics and Astronautics at Stanford University has developed a series of high-order numerical schemes and computational tools that have demonstrated the viability of these schemes. In this paper, a new code named HiFiLES, developed in the ACL and built on top of SD++ (Castonguay et al.\cite{castonguay2011}), is described in detail with a particular emphasis on robustness in a range of applications and V\&V. HiFiLES takes advantage of the synergies between applied mathematics, aerospace engineering, and computer science in order to achieve the ultimate goal of developing an advanced high-fidelity simulation environment.

In addition to the original characteristics of the SD++ code, HiFiLES includes some important physical models and computational methods such as: LES using explicit filters and advanced subgrid-scale (SGS) models, high-order stabilization techniques, shock detection and capturing for compressible flow calculations, convergence acceleration methodologies like p-multigrid, and local and dual time stepping. Some of these techniques will be described in this or related papers.

During the development of this software, several key decisions have been made to guarantee a flexible and lasting infrastructure for industrial Computational Fluid Dynamics simulations:
\begin{itemize}
\item The selection of the Energy-Stable Flux Reconstruction (ESFR) scheme on unstructured grids. The flexibility of this method has been critical to guarantee a correct solution independently of the particular physical characteristics of the problem.
\item High performance, materialized in a multi-GPU implementation that takes advantage of the ease of parallelization afforded by discontinuous solution representation. Furthermore, HiFiLES aims to guarantee compatibility with future vector machines and revolutionary hardware technologies.
\item Code portability by using ANSI C++ and relying on widely-available, and well-supported mathematical libraries like Blas, LAPACK, CuBLAS and ParMetis.
\item Object oriented structure to boost the re-usability and encapsulation of the code. This abstraction enables modifications without incorrectly affecting other portions of the code. Although some level of performance is traded for re-usability and encapsulation, the loss in performance is minor.
\end{itemize}

As the mathematical basis and computational implementation of HiFiLES have been described in previous work~\cite{castonguay2011}, the goal of this paper is to illustrate the level of robustness of HiFiLES for interesting problems. This will be accomplished via a verification and validation study, which is fundamental for increasing the credibility of this technology in a competitive industrial framework.

In particular, to ensure that the implementation of the aforementioned features in HiFiLES is correct, the following verification tests are shown: checks of spatial order of accuracy using the Method of Manufactured Solutions (MMS) in 2D and 3D for viscous  flows in unstructured grids. After the Verification, a detailed Validation of the code is presented to illustrate that the solutions provided by HiFiLES are an accurate representation of the real world. Simulations of complex flows are validated against experimental or Direct Numerical Simulation (DNS) results for the following cases: laminar flat-plane, flow around a circular cylinder, SD7003 wing-section and airfoil at 4$\degr$ angle of attack, the Taylor-Green Vortex, and LES of a square cylinder.

The organization of this paper is as follows. Section~\ref{sec:govEq} provides a description of the governing equations. Section~\ref{sec:numerics} describes the mathematical and numerical algorithms implemented in the code. Section~\ref{sec:verification} focuses on the V \& V of HiFiLES, and finally, the conclusions are summarized in Section~\ref{sec:conclusion}.

Finally, it is our intent for this paper to be the main reference for work that uses or enhances the capabilities of HiFiLES, and for it to serve as a sort of reference for researchers and engineers that would like to develop or implement high-order numerical schemes based on an Energy-Stable Flux Reconstruction (ESFR) approach.
