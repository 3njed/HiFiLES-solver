\section{Flux Reconstruction Approach: Inviscid case}

\noindent Consider solving the following 1D scalar conservation law
\begin{equation}
\frac{\partial u}{\partial t}+\frac{\partial f}{\partial x}=0
\label{governing}
\end{equation}
within an arbitrary periodic domain $\Omegabold$, where $x$ is a spatial coordinate, $t$ is time, $u=u(x,t)$ is a conserved scalar quantity and $f=f(u)$ is the flux of $u$ in the $x$ direction. Further, consider partitioning $\Omegabold$ into $N$ distinct elements each denoted $\Omegabold_n=\{x|x_n<x<x_{n+1}\}$ such that
\begin{equation}
\Omegabold=\bigcup_{n=0}^{N-1}\Omegabold_n,\hspace{1cm}\bigcap_{n=0}^{N-1}\Omegabold_n=\emptyset.
\end{equation}
Finally, having partitioned $\Omegabold$ into separate elements, consider representing the exact solution $u$ within each $\Omegabold_n$ by a polynomial of degree k denoted $u^{\delta D}_n=u^{\delta D}_n(x,t)$ (which is in general piecewise discontinuous between elements), and the exact flux $f$ within each $\Omegabold_n$ by a polynomial of degree k+1 denoted $f^{\delta}_n=f^{\delta}_n(x,t)$ (which is piecewise continuous between elements), such that a total approximate solution $u^{\delta D}=u^{\delta D}(x,t)$ and a total approximate flux $f^{\delta}=f^{\delta}(x,t)$ can be defined within $\Omegabold$ as
\begin{equation}
u^{\delta D}=\bigoplus_{n=0}^{N-1}u_n^{\delta D}\approx u,\hspace{1cm}f^{\delta}=\bigoplus_{n=0}^{N-1}f_n^{\delta}\approx f. 
\end{equation}

\vspace{0.1 in}
\noindent From an implementation perspective, it is advantageous to consider transforming each $\Omegabold_n$ to a standard element $\Omegabold_S=\{r|-1\leq r\leq 1\}$ via the mapping
\begin{equation}
r=\Gamma_n(x)=2\left(\frac{x-x_n}{x_{n+1}-x_n}\right)-1,
\end{equation}
which has the inverse
\begin{equation}
x=\Gamma_n^{-1}(r)=\left(\frac{1-r}{2}\right)x_n+\left(\frac{1+r}{2}\right)x_{n+1}. 
\end{equation}

\vspace{0.1 in}
\noindent Having performed such a transformation, the evolution of $u_n^{\delta D}$ within any individual $\Omegabold_n$ (and thus the evolution of $u^{\delta}$ within $\Omegabold$) can be determined by solving the following transformed equation within the standard element $\Omegabold_S$
\begin{equation}
\frac{\partial\hat{u}^{\delta D}}{\partial t}+\frac{\partial\hat{f}^{\delta}}{\partial r}=0,
\label{stand_ele_govern}
\end{equation}
where
\begin{equation}
\hat{u}^{\delta D}=\hat{u}^{\delta D}(r,t)=u^{\delta D}_n(\Gamma_n^{-1}(r),t),
\end{equation}
is a polynomial of degree $k$,
\begin{equation}
\hat{f}^{\delta}=\hat{f}^{\delta}(r,t)=\frac{f^{\delta}_n(\Gamma_n^{-1}(r),t)}{J_n} ,
\end{equation}

\vspace{0.1 in}
\noindent is a polynomial of degree $k+1$, and $J_n=(x_{n+1}-x_{n})/2$. 

\vspace{0.2 in}
\noindent The FR approach to solving Eq. \eqref{stand_ele_govern} within the standard element $\Omegabold_S$ consists of five stages. The first stage is to define a specific form for $\hat{u}^{\delta D}$. To this end, it is assumed that values of $\hat{u}^{\delta D}$ are known at a set of $k+1$ solution points inside $\Omegabold_S$, with each point located at a distinct position $r_i$ ($i=0$ to $k$). Lagrange polynomials $l_i=l_i(r)$ defined as
\begin{equation}
l_i=\prod_{j=0, j\neq i}^{k}\left(\frac{r-r_j}{r_i-r_j}\right),
\end{equation}
can then be used to construct the following expression for
$\hat{u}^{\delta D}$
\begin{equation}
\hat{u}^{\delta D}=\sum_{i=0}^{k}\hat{u}^{\delta D}_{i}\;l_i,
\label{trans_soln}
\end{equation}
where $\hat{u}^{\delta D}_{i}=\hat{u}^{\delta D}_{i}(t)$ are the known values of $\hat{u}^{\delta D}$ at the solution points $r_i$.

\vspace{0.1 in}
\noindent The second stage of the FR approach involves constructing a degree $k$ polynomial $\hat{f}^{\delta D}=\hat{f}^{\delta D}(r,t)$, defined as the approximate transformed discontinuous flux within $\Omegabold_S$. A collocation projection at the $k+1$ solution points is employed to obtain $\hat{f}^{\delta D}$, which can hence be expressed as
\begin{equation}
\hat{f}^{\delta D}=\sum_{i=0}^{k}\hat{f}^{\delta D}_{i}\;l_i
\end{equation}
where the coefficients $\hat{f}^{\delta D}_{i}=\hat{f}^{\delta D}_{i}(t)$ are simply values of the transformed flux at each solution point $r_i$ evaluated directly from the approximate solution. The flux $\hat{f}^{\delta D}$ is termed discontinuous since it is calculated directly from the approximate solution, which is in general piecewise discontinuous between elements.

\vspace{0.1 in}
\noindent The third stage of the FR approach involves calculating numerical interface fluxes at either end of the standard element $\Omegabold_S$ (at $r=\pm 1$). In order to calculate these fluxes, one must first obtain values for the approximate solution at either end of the standard element via Eq. \eqref{trans_soln}. Once these values have been obtained they can be used in conjunction with analogous information from adjoining elements to calculate numerical interface fluxes. The exact methodology for calculating such numerical interface fluxes will depend on the nature of the equations being solved. For example, when solving the Euler equations one may use a Roe type approximate Riemann solver \cite{Roe81}, or any other two-point flux formula that provides for an upwind bias. In what follows the numerical interface fluxes associated with the left and right hand ends of $\Omegabold_S$ (and transformed appropriately for use in $\Omegabold_S$) will be denoted $\hat{f}^{\delta I}_L$ and $\hat{f}^{\delta I}_R$ respectively.

\vspace{0.1 in}
\noindent The penultimate stage of the FR approach involves adding a degree $k+1$ transformed correction flux $\hat{f}^{\delta C}=\hat{f}^{\delta C}(r,t)$ to the approximate transformed discontinuous flux $\hat{f}^{\delta D}$, such that their sum equals the transformed numerical interface flux at $r=\pm 1$, yet follows (in some sense) the approximate discontinuous flux within the interior of $\Omegabold_S$. In order to define $\hat{f}^{\delta C}$ such that it satisfies the above requirements, consider first defining degree $k+1$ correction functions $g_L=g_L(r)$ and $g_R=g_R(r)$ that approximate zero (in some sense) within $\Omegabold_S$, as well as satisfying 
\begin{equation}
g_L(-1)=1,\hspace{0.5cm}g_L(1)=0,
\label{gL_BCs}
\end{equation}
\begin{equation}
g_R(-1)=0,\hspace{0.5cm}g_R(1)=1,
\end{equation}
and, based on symmetry considerations
\begin{equation}
g_L(r)=g_R(-r).
\end{equation}

\vspace{0.1 in}
\noindent A suitable expression for $\hat{f}^{\delta C}$ can now be written in terms of $g_L$ and $g_R$ as
\begin{equation}
\hat{f}^{\delta C}=(\hat{f}^{\delta I}_L-\hat{f}^{\delta D}_L)g_L+(\hat{f}^{\delta I}_R-\hat{f}^{\delta D}_R)g_R, 
\end{equation}

\vspace{0.05 in}
\noindent where $\hat{f}^{\delta D}_L=\hat{f}^{\delta D}(-1,t)$ and $\hat{f}^{\delta D}_R=\hat{f}^{\delta D}(1,t)$. Using this expression, a degree $k+1$ approximate total transformed flux $\hat{f}^{\delta}=\hat{f}^{\delta}(r,t)$ within $\Omegabold_S$ can be constructed from the discontinuous and correction fluxes as follows
\begin{equation}
\hat{f}^{\delta}=\hat{f}^{\delta D}+\hat{f}^{\delta C}=\hat{f}^{\delta D}+(\hat{f}^{\delta I}_L-\hat{f}^{\delta D}_L)g_L+(\hat{f}^{\delta I}_R-\hat{f}^{\delta D}_R)g_R.
\label{f_total}
\end{equation}

\noindent The final stage of the FR approach involves calculating the divergence of $\hat{f}^{\delta}$ at each solution point $r_i$ using the expression
\begin{equation}
\frac{\partial\hat{f}^{\delta}}{\partial r}(r_i)=\sum_{j=0}^{k}\hat{f}^{\delta D}_j\;\frac{\mathrm{d}l_j}{\mathrm{d}r}(r_i)+(\hat{f}^{\delta I}_L-\hat{f}^{\delta D}_L)\frac{\mathrm{d}g_{L}}{\mathrm{d}r}(r_i)+(\hat{f}^{\delta I}_R-\hat{f}^{\delta D}_R)\frac{\mathrm{d}g_{R}}{\mathrm{d}r}(r_i).
\end{equation}
These values can then be used to advance the approximate transformed solution $\hat{u}^{\delta D}$ in time via a suitable temporal discretization of the following semi-discrete expression
\begin{equation}
\frac{\mathrm{d}\hat{u}^{\delta D}_i}{\mathrm{d}t}=-\frac{\partial\hat{f}^{\delta}}{\partial r}(r_i).
\label{semi_disc}
\end{equation}

\noindent The nature of a particular FR scheme depends solely on three factors, namely the location of the solution collocation points $r_i$, the methodology for calculating the transformed numerical interface fluxes $\hat{f}^{\delta I}_L$ and $\hat{f}^{\delta I}_R$, and finally the form of the flux correction functions $g_L$ (and thus $g_R$). It has been shown previously that a collocation based (under-integrated) nodal DG scheme is recovered in 1D if the corrections functions $g_L$ and $g_R$ are the right and left Radau polynomials respectively \cite{Huynh07}. Also, it has been shown that SD type methods can be recovered (at least for a linear flux function) if the corrections $g_L$ and $g_R$ are set to zero at a set of k points within $\Omegabold_S$ (located symmetrically about the origin) \cite{Jameson10}. Several additional forms of $g_L$ (and thus $g_R$) have also been suggested, leading to the development of new schemes, with various stability and accuracy properties. For further details of these new schemes see the article by Huynh \cite{Huynh07}.

\section{Energy Stable Flux Reconstruction}

\noindent Energy Stable Flux Reconstruction (ESFR) schemes can be recovered if the left and right corrections functions $g_L$ and $g_R$ respectively are defined as
\begin{equation}
g_L=\frac{(-1)^{k}}{2}\left[L_{k}-\left(\frac{\eta_{k}L_{k-1}+L_{k+1}}{1+\eta_k}\right)\right],
\label{final_left}
\end{equation}
and
\begin{equation}
g_R=\frac{1}{2}\left[L_{k}+\left(\frac{\eta_{k}L_{k-1}+L_{k+1}}{1+\eta_k}\right)\right],
\label{final_right}
\end{equation}
where
\begin{equation}
\eta_k=\frac{c(2k+1)(a_kk!)^2}{2},\hspace{1cm}a_{k}=\frac{(2k)!}{2^{k}(k!)^2},
\end{equation}
$L_k$ is a Legendre polynomial of degree $k$, and $c$ is a free scalar parameter that must lie within the range
\begin{equation}
\frac{-2}{(2k+1)(a_kk!)^2}<c<\infty.
\label{c_range}
\end{equation}

\noindent The value of $c$ (as constrained above) ensures the linear stability of the scheme. Details are discussed in \cite{Vincent10}. Particular values of $c$ yield popular higher order schemes (including nodal DG and variants of SD). ESFR schemes have been successfully applied to a range of linear advection problems \cite{Vincent10}. The purpose of this paper is to extend these schemes to handle viscous phenomenon. 