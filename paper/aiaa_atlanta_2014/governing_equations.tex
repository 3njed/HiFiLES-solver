% !TEX root = ./main.tex

\section{Governing Equations}
\label{sec:govEq}
\subsection{Navier Stokes (NS) Equations}
The main purpose of HiFiLES is to perform High-Fidelity Large-Eddy Simulations --hence the name. The form of the NS Equations most useful for understanding how the FR methodology can be used to solve them is the following
\begin{equation}
\frac{\partial U}{\partial t} +  \nabla \cdot {\bf F} = 0
\end{equation}

where ${\bf F} = (F,G,H) = (F_I,G_I,H_I) - (F_V,G_V,H_V)$ and
\begin{equation}
U = \l(
\begin{tabular}{c}
$\rho$\\
$\rho u$\\
$\rho v$\\
$\rho w$\\
$\rho e$
\end{tabular}
\r) \;\; 
F_I = \l(
\begin{tabular}{c}
$\rho u$\\
$\rho u^2 + p$\\
$\rho uv$\\
$\rho uw$\\
$\rho ue + pu$
\end{tabular}
\r) \;\; 
G_I = \l(
\begin{tabular}{c}
$\rho v$\\
$\rho vu$\\
$\rho v^2 + p$\\
$\rho vw$\\
$\rho ve+pv$
\end{tabular}
\r) \;\; 
H_I = \l(
\begin{tabular}{c}
$\rho w$\\
$\rho wu$\\
$\rho wv$\\
$\rho w^2 + p$\\
$\rho we + pw$
\end{tabular}
\r) \;\; 
\end{equation}
\begin{equation}
F_V = \l(
\begin{tabular}{c}
$0$\\
$\sigma_{xx}$\\
$\sigma_{xy}$\\
$\sigma_{xz}$\\
$u_i\sigma_{ix}-q_x$
\end{tabular}
\r) \;\; 
G_V = \l(
\begin{tabular}{c}
$0$\\
$\sigma_{yx}$\\
$\sigma_{yy}$\\
$\sigma_{yz}$\\
$u_i\sigma_{iy}-q_y$
\end{tabular}
\r) \;\; 
H_V = \l(
\begin{tabular}{c}
$0$\\
$\sigma_{zx}$\\
$\sigma_{zy}$\\
$\sigma_{zz}$\\
$u_i\sigma_{iz}-q_z$
\end{tabular}
\r) \;\; 
\end{equation}

As usual, $\rho$ is density, $u$, $v$, $w$ are the velocity components in the $x, y, z$ directions, respectively, and $e$ is total energy per unit mass. In HiFiLES, the pressure is determined from the ideal gas equation of state
\begin{equation}
p = (\gamma - 1)\rho\l(e - \frac{1}{2}\l(u^2 + v^2 + w^2\r)\r)
\end{equation}
the viscous stresses are those of a Newtonian fluid
\begin{equation}
\sigma_{ij} = \mu\l( \frac{\partial u_i}{\partial x_j}
+ \frac{\partial u_j}{\partial x_i} \r)
- \frac{2}{3}\mu \delta_{ij}\frac{\partial u_k}{\partial x_k}
\end{equation}
and the heat fluxes are defined as
\begin{equation}
q_i = -k \frac{\partial T}{\partial x_i}
\end{equation}
where
\begin{equation}
k = \frac{C_p \mu}{\text{Pr}} , \;\; T = \frac{p}{R \rho}
\end{equation}

Pr is the Prandtl number, $C_p$ is the specific heat at constant pressure and $R$ is the gas constant. In the case of air, $\gamma = 1.4$ and Pr $= 0.72$. The dynamic viscosity $\mu$ in HiFiLES can be a constant or a function of temperature using Sutherland's law.


\subsection{Reynolds Averaged Navier-Stokes (RANS) equations}
The compressible NS equations can be used to solve a variety of different flow physics problems but for turbulent flows, direct numerical simulation using these equations can become excessively expensive. For engineering applications, it is customary to perform a Favre averaging procedure to the NS equations to solve a turbulent mean quantity. This leads to a variety of terms which must be modeled in order to provide closure to the resulting RANS equations \cite{wilcox1998turbulence,oliver2008high}. For example, using the one equation Spalart-Allmaras (SA) turbulence model, the conservative form of the RANS equations is very similar to the NS equations with an added equation and source term,
\begin{align}
	\frac{\partial U}{\partial t} +  \nabla \cdot {\bf F} = S
\end{align}

where, 
\begin{equation}
U = \l(
\begin{tabular}{c}
$\rho$\\
$\rho u$\\
$\rho v$\\
$\rho w$\\
$\rho e$ \\
$\rho\tilde\nu$
\end{tabular}
\r) \;\; 
F_I = \l(
\begin{tabular}{c}
$\rho u$\\
$\rho u^2 + p$\\
$\rho uv$\\
$\rho uw$\\
$\rho ue + pu$\\
$\rho u\tilde\nu$
\end{tabular}
\r) \;\; 
G_I = \l(
\begin{tabular}{c}
$\rho v$\\
$\rho vu$\\
$\rho v^2 + p$\\
$\rho vw$\\
$\rho ve+pv$\\
$\rho v\tilde\nu$
\end{tabular}
\r) \;\; 
H_I = \l(
\begin{tabular}{c}
$\rho w$\\
$\rho wu$\\
$\rho wv$\\
$\rho w^2 + p$\\
$\rho we + pw$\\
$\rho w\tilde\nu$
\end{tabular}
\r) \;\; 
\end{equation}
\begin{equation}
F_V = \l(
\begin{tabular}{c}
$0$\\
$\sigma_{xx}$\\
$\sigma_{xy}$\\
$\sigma_{xz}$\\
$u_i\sigma_{ix}-q_x$\\
$\frac{1}{\sigma}(\mu + \mu \psi)\frac{\partial \tilde \nu}{\partial x}$
\end{tabular}
\r) \;\; 
G_V = \l(
\begin{tabular}{c}
$0$\\
$\sigma_{yx}$\\
$\sigma_{yy}$\\
$\sigma_{yz}$\\
$u_i\sigma_{iy}-q_y$\\
$\frac{1}{\sigma}(\mu + \mu \psi)\frac{\partial \tilde \nu}{\partial y}$
\end{tabular}
\r) \;\; 
H_V = \l(
\begin{tabular}{c}
$0$\\
$\sigma_{zx}$\\
$\sigma_{zy}$\\
$\sigma_{zz}$\\
$u_i\sigma_{iz}-q_z$\\
$\frac{1}{\sigma}(\mu + \mu \psi)\frac{\partial \tilde \nu}{\partial z}$
\end{tabular}
\r) \;\; 
\end{equation}
\begin{equation}
S = \l(
\begin{tabular}{c}
$0$\\
$0$\\
$0$\\
$0$\\
$0$ \\
$c_{b_1}\tilde S \rho\nu\psi + \frac{1}{\sigma}\left[c_{b_2}\rho\nabla\tilde\nu\cdot\nabla\tilde\nu\right] - c_{w_1}\rho f_w \left(\frac{\nu\psi}{d}\right)^2$
\end{tabular}
\r) \;\; 
\end{equation}

Note that the flow variables have been redefined as averaged quantities. Also, the viscous stresses now include the Boussinesq approximated Reynolds stress terms,
\begin{equation}
\sigma_{ij} = (\mu+\mu_t)\l( \frac{\partial u_i}{\partial x_j}
+ \frac{\partial u_j}{\partial x_i} \r)
- \frac{2}{3}(\mu+\mu_t) \delta_{ij}\frac{\partial u_k}{\partial x_k}
\end{equation}
and the heat fluxes are redefined as
\begin{equation}
q_i = -C_p\l( \frac{\mu}{\text{Pr}} + \frac{\mu_t}{\text{Pr}_t}\r)\frac{\partial T}{\partial x_i}
\end{equation}
where $\mu_t$ is the dynamic eddy viscosity and $\text{Pr}_t$ is the turbulent Prandtl number. The various terms added by the one equation SA turbulence model are defined in a later section.